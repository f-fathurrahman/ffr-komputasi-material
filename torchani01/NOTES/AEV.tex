\section{Atomic environment vector}

Using Behler-Parrinello symmetry functions (BPSF)

The original BPSF are used to compute an atomic environment vector (AEV)
$\mathbf{G}^{X}_{i}$ for $i$-th atom of a molecule with atomic number $X$
which are composed of $M$ elements:
\begin{equation}
\mathbf{G}^{X}_{i} = \left\{
G_{1}, G_{2}, G_{3}, \ldots, G_{M}
\right\}
\end{equation}

AEVs probe specific regions of an individual atoms's radial and angular chemical
environment.

Each $\mathbf{G}^{X}_{i}$ is then used as input into a single NNP.
Total energy of a molecule, $E_{T}$ is computed from the outputs, $E_{i}$
of the atomic number specific NNPs using:
\begin{equation}
E_{T} = \sum_{i}^{\text{all atoms}} E_{i}
\end{equation}

Radial elements of $\mathbf{G}^{X}_{i}$:
\begin{equation}
G^{\mathrm{R}}_{m} = \sum_{j \neq i} \exp\left[ -\eta
  \left( R_{ij} - R_{\mathrm{shift}} \right)^2 \right]
  f_{\mathrm{C}}\left(  R_{ij} \right)
\end{equation}
with piecewise cutoff function:
\begin{equation}
f_{\mathrm{C}}\left(R_{ij}\right) = \begin{cases}
0.5\cos\left( \dfrac{\pi R_{ij}}{R_{\mathrm{C}}} \right) + 0.5 & \text{for }
R_{ij} \leq R_{\mathrm{C}} \\
0 & \text{for } R_{ij} > R_{\mathrm{C}}
\end{cases}
\end{equation}

The index $m$ is over a set of $\eta$ and $R_{\mathrm{s}}$.
Parameter $\eta$ is used to change the width of the Gaussian distribution
while the purpose of $R_{\mathrm{s}}$ is to shift the center of the peak.
In an ANI potential, only single $\eta$ is used to produce thin Gaussian peaks
and multiple $R_{\mathrm{s}}$ are used to probe outward from the atomic center.
The reasoning behind this specific use of
parameters is two-fold:
\begin{itemize}
\item firstly, when probing with many small $\eta$ parameters,
vector elements can grow to very large values,
which are detrimental to the training of NNPs.
\item Secondly, using $R_{\mathrm{s}}$ in this manner allows the
probing of very specific regions of
the radial environment, which helps with transferability
\end{itemize}

Two modifcations are made to the original version of Behler
and Parrinello's angular symmetry function to produce one
better suited to probing the local angular environment of
complex chemical systems.
\begin{itemize}
\item The first addition is $\theta_{\mathrm{s}}$, which allows
an arbitrary number of shifts in the angular environment, and
\item the second is a modified exponential factor that allows an
$R_{\mathrm{s}}$ parameter to be added.
The $R_{\mathrm{s}}$ addition allows the angular
environment to be considered within radial shells based on the
average of the distance from the neighboring atoms.
\end{itemize}

The effect
of these two changes is that AEV elements are generally smaller
because they overlap atoms in different angular regions less and
they provide a distinctive image of various molecular features,
a property that assists neural networks in learning the energetics
of specific bonding patterns, ring patterns, functional groups, or
other molecular features.

Given atoms $i$, $j$, and $k$ angle $\theta_{ijk}$, centered on atom $i$,
is computed along with two distances $R_{ij}$ and $R_{ik}$.

A single element $G_{m}^{\mathrm{Amod}}$ of $\mathbf{G}_{i}^{\mathrm{X}}$
to probe the angular environment of atom $i$, takes the form of a sum over
all $j$ and $k$ neighboring atom pairs, of the product of a radial and an
angular factor
\begin{equation}
G_{m}^{\mathrm{Amod}} = 2^{1-\zeta}
\sum_{j,k \neq i}^{\text{all atoms}}
\left( 1 + \cos(\theta_{ijk} - \theta_{\mathrm{s}}) \right)^{\zeta}
\times
\exp\left[ -\eta \left( \dfrac{R_{ij} + R_{jk}}{2} -
R_{\mathrm{s}} \right)^2 \right]
f_{\mathrm{C}}(R_{ij}) f_{\mathrm{C}}(R_{ik})
\end{equation}

The Gaussian factor combined with the cutoff functions, like
the radial symmetry functions, allows chemical locality to be
exploited in the angular symmetry functions.
In this case, the index $m$ is over four separate
parameters: $\zeta$, $\theta_{s}$,
$h$, and $R_{\mathrm{s}}$.
$\eta$ and $R_{\mathrm{s}}$ serve a similar purpose as in eqn (3).
Applying a $\theta_{\mathrm{s}}$ parameter
allows probing of specific regions of the angular environment in
a similar way as is accomplished with $R_{\mathrm{s}}$ in the radial part.
Also, $z$ changes the width of the peaks in the angular environment
