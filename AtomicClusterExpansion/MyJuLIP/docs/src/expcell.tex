
\documentclass[11pt,a4paper]{amsart}


\begin{document}

\begin{center}
   \bf Note on the Exponential Cell Matrix
\end{center}

\bigskip

This brief note is to recap the deformation of a cell matrix via the matrix
exponential (Lie group). First of all, we don't work in terms of the cell matrix
$C$, but in terms of its transpose $A = C^T$. We will then write
%
\[
  A = \exp(U) A_0,
\]
where $A_0$ is some reference cell matrix.

\subsection*{Virial and Stress}
%
First, however, we recap the definition of virial and of stress. Let $R$ be the
current positions and $A = C^T$ the current cell matrix. When we deform the cell
matrix, then the positions deform accordingly,
%
\[
   A \mapsto F \cdot A \quad \Rightarrow \quad R \mapsto F \cdot R.
\]
The (negative) virial is the derivative of the PES with respect to $F$
under this transformation. That is, if $E(A, R)$ denotes the PES with
cell matrix $A$ and positions $R$ then
\[
   - T^{\rm vir}_{i\alpha} := \frac{\partial}{\partial F_{i\alpha}}
      E( F A, F R ) \Big|_{F = I}.
\]
For example, for a pair potential,
\[
   E = \frac12 \sum_{a \neq b} \phi(r_{ab}),
\]
with this definition we obtain
\[
   T^{\rm vir} = - \frac12 \sum_{a \neq b} \phi'(r_{ab}) R_{ab} \otimes R_{ab},
\]
which should be the correct definition. (For EAM the expression is analogous, in general it can be a bit more complex.)

We can connect this to the Piola--Kirchhoff and Cauchy stresses as follows: Let
$A_0$ be some reference cell matrix, and $R_0$ the reference cell positions and
let $A = F A_0, R = F R_0$, then the Cauchy--Born energy per unit (underformed)
volume is
\[
   W(F) = (\det A_0)^{-1} E(F A_0, F R_0)
\]
and its derivative $T^{\rm pk} := \partial_F W(F)$ is the first Piola--Kirchhoff
stress. For a pair potential this can be readily calculated as
%
\begin{equation} \label{eq:Tpk-pair}
   T^{\rm pk} = \frac12 \sum_{a \neq b} \phi'(r_{ab}) R_{ab} \otimes R_{0,ab};
\end{equation}
%
(i.e., $r_{ab} = |F R_{0,ab}|$ and $R_{ab} = F R_{0,ab}$) note in particular
that this is asymmetric.

The Cauchy--Stress is given by
\[
   T^{\rm pk} = T^{\rm c} {\rm cof} F,
\]
where ${\rm cof} F = {\rm det F} F^{-T}$,  that is,
\[
   T^{\rm c} = (\det F)^{-1} T^{\rm pk} F^T.
\]
%
Returning to the example of a pair potential, inserting this expression into
\eqref{eq:Tpk-pair}, recalling $A = F A_0, R = F R_0$ and using $\det A_0 \det F
= \det (F A_0) = \det A$,  we obtain
%
\begin{align*}
   T^{\rm c} &= (\det A_0 \det F)^{-1} \sum_{a \neq b} \phi'(r_{ab}) (F R_{0,ab}) \otimes (F R_{0,ab})  \\
   &= (\det A)^{-1} \sum_{a \neq b} \phi'(r_{ab}) R_{ab} \otimes R_{ab} \\
   &= - (\det A)^{-1} T^{\rm vir}.
\end{align*}
%
In general, we take this as the definition of atomistic (Cauchy) stress:
%
\begin{equation} \label{eq:defn-at-stress}
   T^{\rm c} := - (\det A)^{-1} T^{\rm vir},
\end{equation}
i.e., negative virial normalised with respect to deformed cell volume.

\subsection*{Cell Deformation via Exponential Map}
%
We are now simply replacing the cell deformation  $A_0 \mapsto F A_0$ with $A_0
\mapsto \exp(U) A_0$. More generally, we allow a cell matrix $A = \exp(U) A_0$
and positions $R = \exp(U) R_0$ and consider the PES $E$ as a function of $U$
and $R$, hence we need to compute
%
\begin{align*}
   T^{\rm log}_{i\alpha}
   &:=  \frac{\partial}{\partial U_{i\alpha}} E\big( \exp(U) A_0, \exp(U) R_0 \big) \\
   &= \frac{\partial}{\partial V_{i\alpha}} E\big( \exp(V) A, \exp(V) R \big) \Big|_{V = 0}.
\end{align*}
There are several ways to derive an expression, we will first start from the
virial.

Let $L(U, V) := \partial \exp(U) : V$ be the linearisation of the exponential
map, that is,
\[
   L(U, V) := \frac{d}{dt} \exp(U + t V) \Big|_{t = 0},
\]
then writing $F = \exp(V)$ and applying the chain rule we clearly have
%
\begin{align*}
   T^{\rm log} : V = T^{\rm log}_{i\alpha} V_{i\alpha}
   &=
   \big(-T^{\rm vir}_{j\beta}\big)\,
   \frac{\partial F_{j\beta}}{\partial V_{i\alpha}} |_{V=0} V_{i\alpha} \\
   &=
   - T^{\rm vir}_{j\beta} \, \big[L(0, V) \big]_{j\beta}
   = - T^{\rm vir} : L(0, V).
\end{align*}
Since $\exp(V) = I + V + O(V^2)$ it follows that $L(0, V) = V$, and hence we
simply have
\[
   T^{\rm log} = - T^{\rm vir}.
\]
We could have just guessed this, since for infinitesimal $V$ we just
have $\exp(V) \sim I + V$. The exponential map becomes important
only when we allow large deformations.

\subsection*{Geometry Optimisation}
%
In geometry optimisation it is not always convenient to compute gradients in the
deformed configuration, but to fit within the standard optimisation frameworks
it can be expedient to freeze some reference configuration and represent the
deformed configuration in terms of that reference. In {\tt JuLIP} this is done
as follows (but other choices are possible and may indeed be better!):
%
\begin{itemize}
   \item At the beginning of the optimisation a reference cell $A_0$ and reference positions $R_0$ are frozen.
   \item Given a current configuration $A, R$ we represent it
   as
   \[
      A = \exp(U) A_0 \qquad \text{and} \qquad
      R = \exp(U) X
   \]
   \item We take $(U, X)$ as the degrees of freedom with respect to which we
   optimise. We also enforce that $U$ is symmetric. Any anti-symmetric component
   corresponds to a rotation and does not change the energy of the system.
\end{itemize}
%
Let $\Phi(U, X) := E(A, R)$ then we need to compute $\nabla_U \Phi, \nabla_X
\Phi$. We begin with the latter: since $R_i = \exp(U) X_i$ we simply have
\[
   \nabla_{X_i} \Phi = \bigg[\frac{\partial R_i}{\partial X_i}\bigg]^T \nabla{R_i} E
      = \exp(U) \nabla_{R_i} E,
\]
where $\nabla_{R_i} E$ is the negative force acting on atom $i$. Here, we also
employed our assumption that $U$ is symmetric.

We now turn towards $\nabla_{U} \Phi$. Note that we have $\Phi(U, X) := E(A, R)
= E(\exp(U) A_0, \exp(U) X)$, that is, computing $\nabla_U \Phi$ is similar to
the computation of the Piola--Kirchhoff stress, except for the volume
normalisation; that is, with $F = \exp(U)$ we have
\begin{align*}
   \nabla_U \Phi : V
   &=
   (\partial_{U_{i\alpha}} \Phi)\, V_{i\alpha} \\
   %
   &=
   \partial_{F_{j\beta}} E(F A_0, F X)
   \frac{\partial F_{j\beta}}{\partial U_{i\alpha}} V_{i\alpha} \\
   %
   &=
   \big[ (\det A_0) T^{\rm pk} \big]_{j\beta} \big[L(U, V)\big]_{j\beta} \\
   %
   &=
   \big[ - T^{\rm vir} \exp(-U) \big]_{j\beta} \big[L(U, V)\big]_{j\beta} \\
   %
   &=
   - \big[ T^{\rm vir} \exp(-U) \big] : L(U, V).
\end{align*}
Finally, it only remains to observe that
\[
   G : L(U, V) = V : L(U, G),
\]
which can be proven directly by manipulating the series representation to obtain
that
\begin{equation} \label{eq:grad_U}
   \nabla_U \Phi = - L(U, T^{\rm vir} \exp(-U)).
\end{equation}

\end{document}
