\documentclass[a4paper,11pt]{extarticle}
\usepackage[a4paper]{geometry}
\geometry{verbose,tmargin=2cm,bmargin=2cm,lmargin=2cm,rmargin=2cm}

%\usepackage{graphicx}

\usepackage{fontspec}
\setmonofont{FreeMono}

\setlength{\parindent}{0cm}
\setlength{\parskip}{0.5em}

\usepackage{textcomp}

\usepackage{hyperref}
\usepackage{url}
\usepackage{xcolor}

\usepackage{minted}
\newminted{cpp}{breaklines,fontsize=\small}
\newminted{gnuplot}{breaklines,fontsize=\small}
\newminted{text}{breaklines,fontsize=\small}

\definecolor{mintedbg}{rgb}{0.95,0.95,0.95}
\usepackage{mdframed}

\BeforeBeginEnvironment{minted}{\begin{mdframed}[backgroundcolor=mintedbg]}
\AfterEndEnvironment{minted}{\end{mdframed}}

\title{Instalasi VASP 5.2}
\author{Fadjar Fathurrahman}
\date{2018}

\begin{document}
\maketitle

Asumsi: instalasi akan dilakukan pada sistem operasi Ubuntu versi 16.04.

\begin{itemize}
\item GNU Fortran compiler
\item BLAS dan LAPACK
\item FFTW3
\item OpenMPI
\end{itemize}

Direktori:
\begin{itemize}
\item \texttt{vasp.5.lib}
\item \texttt{vasp.5.2}
\end{itemize}

Kompilasi \texttt{vasp.5.lib}:
\begin{textcode}
make -f makefile.linux_gfortran
\end{textcode}

\end{document}
