\documentclass[a4paper,10pt,bahasa]{paper}

\usepackage[a4paper]{geometry}
\geometry{verbose,tmargin=1.5cm,bmargin=1.5cm,lmargin=1.5cm,rmargin=1.5cm}

\setlength{\parskip}{\smallskipamount}
\setlength{\parindent}{0pt}

\usepackage{hyperref}
\usepackage{url}
\usepackage{xcolor}

\usepackage{amsmath}
\usepackage{amssymb}
\usepackage{braket}

\usepackage{minted}
\newminted{julia}{breaklines}
\newminted{text}{breaklines}

\newcommand{\jlcode}[1]{\mintinline{julia}{#1}}

\usepackage{babel}

\begin{document}

\title{PPNN ITB HPC Manual}
\author{Fadjar Fathurrahman}
\maketitle

\tableofcontents

\part{Quick Start}

\section{Login ke HPC PPNN}

Alamat IP publik dari headnode adalah 167.205.6.67.

Anda dapat mengaksesnya melalui SSH jika telah mendapatkan username
dari admin HPC PPNN

\begin{textcode}
ssh username@167.205.6.67
\end{textcode}

atau

\begin{textcode}
ssh -l username 167.205.6.67
\end{textcode}

Silakan menghubungi admin jika menemui kesulitan untuk login.

\part{Tutorial}

\section{Pengenalan \texttt{bash}}

Pengenalan `bash`

Sistem operasi pada HPC PPNN adalah Red Hat Linux.

Perintah-perintah dasar

ls

mkdir

Teks editor

vim

nano

emacs

File .bashrc

Variabel-variabel penting

PATH

\verb|LD_LIBRARY_PATH|








\end{document}
