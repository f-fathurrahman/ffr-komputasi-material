\documentclass[a4paper,11pt]{extarticle}

\usepackage[a4paper]{geometry}
\geometry{verbose,tmargin=2cm,bmargin=2cm,lmargin=2cm,rmargin=2cm}

\usepackage{amsmath} % load this before unicode-math
\usepackage{amssymb}

\usepackage{unicode-math}

\usepackage{fontspec}
\setmainfont{TeX Gyre Termes}
\setmathfont{TeX Gyre Termes Math}
\setmonofont{DejaVu Sans Mono}

\setlength{\parindent}{0cm}
\setlength{\parskip}{0.5em}

\usepackage{hyperref}
\usepackage{url}
\usepackage{xcolor}

\usepackage{minted}
\newminted{python}{breaklines,fontsize=\scriptsize}
\newminted{text}{breaklines,fontsize=\scriptsize}

\newcommand{\txtinline}[1]{\mintinline[fontsize=\scriptsize]{text}{#1}}
\newcommand{\pyinline}[1]{\mintinline[fontsize=\scriptsize]{python}{#1}}

\definecolor{mintedbg}{rgb}{0.95,0.95,0.95}
\usepackage{mdframed}

\BeforeBeginEnvironment{minted}{\begin{mdframed}[backgroundcolor=mintedbg]}
\AfterEndEnvironment{minted}{\end{mdframed}}

\begin{document}

\title{
Modul Praktikum \\
Pengenalan \textsf{ASE} dan \textsf{GPAW}}
\author{Fadjar Fathurrahman\\
Vieri Kristianto Wijaya\\
Eraraya Ricardo Muten}
\date{2019}
\maketitle

\section{Tujuan}
\begin{itemize}
\item Mampu membuat struktur atomistik: xyz, XSF. ase.build.bulk, ase.io.read,
\end{itemize}

\section{Perangkat lunak yang diperlukan}


\section{Pengenalan \textsf{ASE}}

\textsf{ASE} (Atomic Simulation Environment) adalah kumpulan modul dan \textit{script} Python untuk
mempersiapkan, memanipulasi, menjalankan, menggambarkan (visualisasi) and menganalisis simulasi
atomistik.

Kita akan menggunakan \textsf{ASE} untuk membuat struktur atomistik.

Beberapa kelas (tipe data) penting pada \textsf{ASE}: \pyinline{Atoms} dan \pyinline{Calculator}

\begin{pythoncode}
import ase

atoms = bulk("Cu", "fcc")
\end{pythoncode}



\end{document}