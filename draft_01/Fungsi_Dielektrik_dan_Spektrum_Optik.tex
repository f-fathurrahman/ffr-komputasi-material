\documentclass[a4paper,bahasa]{paper}

\usepackage{amssymb}
\usepackage{amsmath}

\usepackage{setspace}
\onehalfspacing

\usepackage{babel}

\renewcommand{\imath}{\mathfrak{i}}
\newcommand{\emath}{\mathrm{e}}

\begin{document}

\title{Fungsi dielektrik dan spektrum optik}
\author{Fadjar Fathurrahman}
\date{}
\maketitle

\section{Pendahuluan}

TODO: definisi fungsi dielektrik

Spektrum optik ?

\section{Fungsi dielektrik dari model padatan}

Dalam bagian ini, konsep fungsi dielektrik akan dikenalkan dengan meninjau
sistem yang paling sederhana, yaitu atom-atom H yang tersusun pada latis kubik
yang akan disebut sebagai \textit{kristal hidrogen padat}.
Pembahasan akan dimulai dari penurunan polarizabilitas elektrik dari sebuah
hidrogen atom, kemudian dengan menggunakan hubungan Clausius-Mosotti akan
diperoleh fungsi dielektrik dari hidrogen padat.

\subsection{Dinamika elektron dalam suatu medan radiasi}

Tinjau sebuah atom hidrogen, dengan inti atom (proton) dijaga tetap pada suatu titik
acuan. Pada waktu $t < 0$ sistem ini berada pada keadaan dasar, yaitu elektron
menempati orbital $1s$, $\phi_{1s}$, dengan energi $E_{1s}$. Pada saat $t > 0$
suatu medan listrik eksternal akan dikenakan pada sistem ini dan efeknya
terhadap atom hidrogen akan dipelajari.
Medan listrik eksternal ini bersifat homogen dan bekerja pada arah sumbu $x$
%
\begin{equation}
\mathbf{E}(t) =
\begin{cases} 
\mathcal{E} \cos(\omega t) \mathbf{u}_{x} & \mbox{untuk } t \geq 0\\
0 & \mbox{jika } t < 0
\end{cases}
\label{eq:medan_E_t}
\end{equation}
%
Persamaan ini mendeskripsikan radiasi elektromagnetik monokromatik
dengan frequensi $\omega$ yang dapat diperoleh, misalnya, dengan menggunakan
laser. Untuk persamaan-persamaan berikutnya, akan digunakan sistem satuan SI.
Menurut teori elektromagnetika klasik, energi potensial dari sebuah elektron
yang terletak pada titik $\mathbf{r}$ di dalam medan ini adalah:
\begin{equation}
V(\mathbf{r},t) = e \mathbf{E}(t) \cdot \mathbf{r}.
\end{equation}
Untuk mempelajari evolusi waktu dari fungsi gelombang, $\psi(\mathbf{r},t)$,
diperlukan solusi dari persamaan Schr\"{o}dinger bergantung waktu:
%
\begin{equation}
\imath \hbar \frac{\partial}{\partial t} \psi(\mathbf{r},t) =
\left[ \hat{H} + V(\mathbf{r},t) \right] \psi(\mathbf{r},t)
\label{eq:td_sch}
\end{equation}
%
Dalam persamaan ini, $\hat{H}$ adalah Hamiltonian dari atom hidrogen
ketika tidak ada medan eksternal
%
\begin{equation}
\hat{H} = -\frac{\hbar^2}{2m_{e}}\nabla^2 -
\frac{e^2}{4\pi\varepsilon_{0}\left|\mathbf{r}\right|}
\end{equation}
%
dan potensial $V$ adalah potensial eksternal, yaitu pada Persamaan \eqref{eq:medan_E_t}.
Kondisi awal dari evolusi fungsi gelombang ini adalah
\begin{equation}
\psi(\mathbf{r}, t < 0) = \phi_{1s}(\mathbf{r})
\end{equation}

Sebagai penyederhanaan,Hamiltonian diasumsikan $H$ hanya menerima solusi dari
keadaan kuantum $1s$, $2s$, $2p_{x}$, $2p_{y}$, dan $2p_{z}$.
Dapat diverifikasi bahwa
%
\begin{equation}
\phi_{i}(\mathbf{r})\exp\left[ -\frac{\imath}{\hbar} E_{i}t \right]
\end{equation}
%
dengan $i = 1s, 2s, \ldots, 2p_{z}$ adalah solusi dari Persamaan \eqref{eq:td_sch}
untuk $t > 0$ ketika $V = 0$.
Kombinasi linear fari fungsi-fungsi tersebut:
%
\begin{equation}
\psi(\mathbf{r},t) =
c_{1s}(t) \psi_{1s}(\mathbf{r}) \exp\left[-\frac{\imath}{\hbar} E_{1s}t\right]
+ \ldots +
c_{2p_{z}}(t) \psi_{2p_{z}}(\mathbf{r}) \exp\left[-\frac{\imath}{\hbar} E_{2p_{z}}t\right]
\label{eq:psi_komb_lin}
\end{equation}
%
juga merupakan solusi dari Persamaan \eqref{eq:td_sch}.
Perhatikan bahwa koefisien $c_{1s}, c_{2s}, \ldots, c_{2p_{z}}$ bergantung pada waktu.
%
Karena dalam sistem ini hanya terdapat satu elektron, syarat normalisasi
\begin{equation}
\int \psi^{*}(\mathbf{r},t)\,\psi(\mathbf{r},t)\,\mathrm{d}\mathbf{r} = 1
\end{equation}
dapat digunakan untuk mendapatkan kondisi berikut:
\begin{equation}
\left| c_{1s}(t) \right|^2 + \left| c_{2s}(t) \right|^2 +
\left| c_{2p_{x}}(t) \right|^2 + \left| c_{2p_{y}}(t) \right|^2 +
\left| c_{2p_{z}}(t) \right|^2
\end{equation}
%
Kondisi ini berguna untuk mendapat gambaran intutif mengenai bentuk dari $\psi_(\mathbf{r},t)$.
Sebagai contoh, apabila kita memiliki $\left|c_{1s}\right|^2 = 0.9$ dan 
$\left|c_{2p_{x}}\right|^2 = 0.1$, maka kita dapat menebak bahwa fungsi gelombang adalah
campuran dari 90\% $1s$ dan 10\% $2p_{x}$.

Bentuk dan energi dari orbital atom hidrogen telah umum diketahui sehingga yang harus
ditentukan dari Persamaan \eqref{eq:psi_komb_lin} adalah koefisien-koefisien
yang bergantung waktu. Mereka dapat ditentukan dengan cara sebagai berikut.
Pertama substitusi Persamaan \eqref{eq:psi_komb_lin} ke dalam Persamaan \eqref{eq:td_sch}.
Kemudian kalikan kedua sisi persamaan yang dihasilkan dengan $\phi_{1s}(\mathbf{r})$
dan lakukan integrasi terhadap variabel ruang. Dengan mengingat bahwa setiap orbital
adalah keadaan eigen dari Hamiltonian $\hat{H}$ dan setiap orbital yang berbeda
adalah ortogonal, untuk $t \geq 0$ dapat diperoleh:
%
\begin{align}
\imath\hbar\frac{\mathrm{d}c_{1s}}{\mathrm{d}t} & =
e\,\mathcal{E}\,x_{1s,1s}\cos(\omega t)\exp\left[\frac{i}{\hbar}(E_{1s} - E_{1s})\right] \\
& + e\,\mathcal{E}\,x_{1s,2s}\cos(\omega t)\exp\left[\frac{i}{\hbar}(E_{1s} - E_{2s})\right] \\
& + e\,\mathcal{E}\,x_{1s,2p_{x}}\cos(\omega t)\exp\left[\frac{i}{\hbar}(E_{1s} - E_{2p_{x}})\right] \\
& + e\,\mathcal{E}\,x_{1s,2p_{y}}\cos(\omega t)\exp\left[\frac{i}{\hbar}(E_{1s} - E_{2p_{y}})\right] \\
& + e\,\mathcal{E}\,x_{1s,2p_{z}}\cos(\omega t)\exp\left[\frac{i}{\hbar}(E_{1s} - E_{2p_{z}})\right]
\end{align}
%
dengan $x_{1s,1s}, x_{1s,2s}, \ldots, x_{1s,2p_{z}}$ adalah matriks element dari
operator posisi $\hat{x}$:
\begin{equation}
x_{i,j} = \int\mathrm{d}\mathbf{r}\,\phi_{i}(\mathbf{r})\,x\,\phi_{j}(\mathbf{r}),\hspace{1em}
i,j = 1s, 2s, \ldots, 2p_{x}
\end{equation}

\end{document}


