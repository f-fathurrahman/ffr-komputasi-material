\section{PW codes how to}

\subsection{Running the code}

\subsubsection{PWSCF}
\begin{textcode}
pw.x < PWINPUT > LOG1
\end{textcodt}

\begin{textcode}
mpirun -n 4 pw.x < PWINPUT > LOG1
\end{textcode}


\subsubsection{ABINIT}

\begin{textcode}
abinit < FILES > LOG1
\end{textcode}

\begin{textcode}
mpirun -n 4 abinit < FILES > LOG1
\end{textcode}


\subsubsection{JDFTX}

\begin{textcode}
jdftx -c 1 -i INPUT > LOG1
\end{textcode}



\subsection{Specifying unit cell}

\subsubsection{PWSCF}

\begin{textcode}
&SYSTEM
  ibrav = 0
/
\end{textcode}

\begin{textcode}
CELL_PARAMETERS bohr
  -5.1315500000   0.0000000000   5.1315500000
   0.0000000000   5.1315500000   5.1315500000
  -5.1315500000   5.1315500000   0.0000000000
\end{textcode}


\subsubsection{ABINIT}

\begin{textcode}
acell 1.0 1.0 1.0
rprim
  -5.1315500000   0.0000000000   5.1315500000
   0.0000000000   5.1315500000   5.1315500000
  -5.1315500000   5.1315500000   0.0000000000
\end{textcode}


\subsubsection{JDFTX}

Example:

\begin{textcode}
lattice \
  3.61496   3.61496   0.00000 \
  3.61496   0.00000   3.61496 \
  0.00000   3.61496   3.61496 
\end{textcode}

The lattice matrix is arranged by column.

The unit is bohr. We can use \verb|latt-scale| to use other units. For example:
\begin{textcode}
latt-scale 1 1 1
\end{textcode}

\subsection{Specifying metallic occupation}

\subsubsection{PWSCF}



\subsubsection{ABINIT}




\subsubsection{JDFTX}

\verb|elec-smearing <smearingType>=Cold|Fermi|Gauss <smearingWidth>|

\begin{itemize}
\item \verb|elec-smearing Fermi 0.001|
\item \verb|elec-smearing Gauss 0.001|
\item \verb|elec-smearing Cold 0.001|
\end{itemize}

\verb|smearingWidth| is given in Hartree.
