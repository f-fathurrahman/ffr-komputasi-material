\documentclass[a4paper]{paper}

\usepackage[a4paper]{geometry}
\geometry{verbose,tmargin=2.0cm,bmargin=2.0cm,lmargin=2.0cm,rmargin=2.0cm}

%\usepackage{fontspec}
%\setmonofont{DejaVu Sans Mono}

\usepackage{hyperref}
\usepackage{url}
\usepackage{xcolor}
\usepackage{minted}

% For DejaVu Sans Mono, use smaller font size
\newminted{fortran}{frame=lines,framesep=2mm,breaklines,fontsize=\footnotesize}
\newminted{text}{breaklines,fontsize=\footnotesize}
\newminted{julia}{breaklines,fontsize=\footnotesize}
\newminted{bash}{breaklines,fontsize=\footnotesize}
\newminted{c}{breaklines,fontsize=\footnotesize}
\newminted{shell}{breaklines,fontsize=\footnotesize}

\setlength{\parskip}{\smallskipamount}
\setlength{\parindent}{0pt}

\begin{document}

\title{A tutorial on practical density functional calculations using plane
wave basis set}

\author{Fadjar Fathurrahman}
\maketitle

\part{Basic Theory}

\section{Introduction}

This is a tutorial explaining basic density functional theory calculations
for (mainly) condensed matter systems using plane wave basis set.

Example equations:
\[
\int_{0}^{\infty}\frac{\alpha}{\beta}\mathrm{d}\Gamma
\]

\section{Atomic structure specification}

We need to specify unit cell and basis (atomic positions within the unit cell)

cif file

xyz file

cif2cell

GUI tools: avogadro

ASE


assumption: the softwares are already installed in your systems.


\part{Silicon crystal: total energy and electronic structure}

A tutorial about input file

\section{Using Quantum ESPRESSO}

Using ONCV PBE pspot
\begin{verbatim}
$ grep ! LOG1 
!    total energy              =     -15.71919101 Ry
\end{verbatim}
or  -7.859595505 Ha

Using GTH pspot and VWN xc
\begin{verbatim}
$ grep ! LOG1 
!    total energy              =     -15.82201747 Ry
\end{verbatim}
or -7.911008735 Ha



\section{Using ABINIT}

Using ONCV PBE pspot
\begin{verbatim}
$ grep Etotal LOG1
    >>>>>>>>> Etotal= -8.43967410558013E+00
\end{verbatim}

Using GTH vwn pspot
\begin{verbatim}
$ grep Etotal LOG1
    >>>>>>>>> Etotal= -7.91100879337138E+00
\end{verbatim}

\section{Using JDFTX}

\begin{verbatim}
jdftx -c 1 < INPUT
\end{verbatim}


\begin{verbatim}
$ grep "Etot =" LOG1 
     Etot =       -7.8595945951064721
\end{verbatim}

\section{PW codes how to}

\subsection{Running the code}

\subsubsection{PWSCF}
\begin{textcode}
pw.x < PWINPUT > LOG1
\end{textcodt}

\begin{textcode}
mpirun -n 4 pw.x < PWINPUT > LOG1
\end{textcode}


\subsubsection{ABINIT}

\begin{textcode}
abinit < FILES > LOG1
\end{textcode}

\begin{textcode}
mpirun -n 4 abinit < FILES > LOG1
\end{textcode}


\subsubsection{JDFTX}

\begin{textcode}
jdftx -c 1 -i INPUT > LOG1
\end{textcode}



\subsection{Specifying unit cell}

\subsubsection{PWSCF}

\begin{textcode}
&SYSTEM
  ibrav = 0
/
\end{textcode}

\begin{textcode}
CELL_PARAMETERS bohr
  -5.1315500000   0.0000000000   5.1315500000
   0.0000000000   5.1315500000   5.1315500000
  -5.1315500000   5.1315500000   0.0000000000
\end{textcode}


\subsubsection{ABINIT}

\begin{textcode}
acell 1.0 1.0 1.0
rprim
  -5.1315500000   0.0000000000   5.1315500000
   0.0000000000   5.1315500000   5.1315500000
  -5.1315500000   5.1315500000   0.0000000000
\end{textcode}


\subsubsection{JDFTX}

Example:

\begin{textcode}
lattice \
  3.61496   3.61496   0.00000 \
  3.61496   0.00000   3.61496 \
  0.00000   3.61496   3.61496 
\end{textcode}

The lattice matrix is arranged by column.

The unit is bohr. We can use \verb|latt-scale| to use other units. For example:
\begin{textcode}
latt-scale 1 1 1
\end{textcode}

\subsection{Specifying metallic occupation}

\subsubsection{PWSCF}



\subsubsection{ABINIT}




\subsubsection{JDFTX}

\verb|elec-smearing <smearingType>=Cold|Fermi|Gauss <smearingWidth>|

\begin{itemize}
\item \verb|elec-smearing Fermi 0.001|
\item \verb|elec-smearing Gauss 0.001|
\item \verb|elec-smearing Cold 0.001|
\end{itemize}

\verb|smearingWidth| is given in Hartree.



\end{document}
