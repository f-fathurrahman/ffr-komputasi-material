\section{Poisson equation}

In this section we will discuss a second equation that is important
in solving Kohn-Sham equation, namely the Poisson equation. In the
context of solving Kohn-Sham equation, Poisson equation is used to
calculate classical electrostatic potential due to some electronic
charge density.
The Poisson equation that we will solve have the following form:
\begin{equation}
\nabla^2 V_{\mathrm{Ha}}(\mathbf{r}) = -4\pi\rho(\mathbf{r})
\label{eq:poisson_3d}
\end{equation}
where $\rho(\mathbf{r})$ is the electronic density. Using finite
difference discretization for the operator $\nabla^2$ we end up with
the following linear equation:
\begin{equation}
\mathbb{L} \mathbf{V} = \mathbf{f}
\label{eq:linear_eq_poisson}
\end{equation}
where $\mathbb{L}$ is the matrix representation of the Laplacian operator
$\mathbf{f}$ is the discrete representation of the right hand side of the equation
\ref{eq:poisson_3d}, and the unknown $\mathbf{V}$ is the discrete representation of
the Hartree potential.


There exist several methods for solving the linear equation \ref{eq:linear_eq_poisson}.
We will use the so-called conjugate gradient method for solving this equation.

The algorithm is described in \txtinline{Poisson_solve_PCG.jl}

\begin{juliacode}
function Poisson_solve_PCG( Lmat::SparseMatrixCSC{Float64,Int64},
                            prec,
                            rho::Array{Float64,1}, NiterMax::Int64;
                            TOL=5.e-10 )
    Npoints = size(rho,1)
    phi = zeros( Float64, Npoints )
    r = zeros( Float64, Npoints )
    p = zeros( Float64, Npoints )
    z = zeros( Float64, Npoints )
    nabla2_phi = Lmat*phi
    r = rho - nabla2_phi
    z = copy(r)
    ldiv!(prec, z)
    p = copy(z)
    rsold = dot( r, z )
    for iter = 1 : NiterMax
        nabla2_phi = Lmat*p
        alpha = rsold/dot( p, nabla2_phi )
        phi = phi + alpha * p
        r = r - alpha * nabla2_phi
        z = copy(r)
        ldiv!(prec, z)
        rsnew = dot(z, r)
        deltars = rsold - rsnew
        if sqrt(abs(rsnew)) < TOL
            break
        end
        p = z + (rsnew/rsold) * p
        rsold = rsnew
    end
    return phi
end
\end{juliacode}

Example problem: (adapted frm Prof. Arias Practical DFT course)
\begin{juliacode}
function test_main( NN::Array{Int64} )
    AA = [0.0, 0.0, 0.0]
    BB = [16.0, 16.0, 16.0]
    # Initialize grid
    FD = FD3dGrid( NN, AA, BB )
    # Box dimensions
    Lx = BB[1] - AA[1]
    Ly = BB[2] - AA[2]
    Lz = BB[3] - AA[3]
    # Center of the box
    x0 = Lx/2.0
    y0 = Ly/2.0
    z0 = Lz/2.0
    # Parameters for two gaussian functions
    sigma1 = 0.75 
    sigma2 = 0.50

    Npoints = FD.Nx * FD.Ny * FD.Nz
    rho = zeros(Float64, Npoints)
    phi = zeros(Float64, Npoints)
    # Initialization of charge density
    dr = zeros(Float64,3)
    for ip in 1:Npoints
        dr[1] = FD.r[1,ip] - x0
        dr[2] = FD.r[2,ip] - y0
        dr[3] = FD.r[3,ip] - z0
        r = norm(dr)
        rho[ip] = exp( -r^2 / (2.0*sigma2^2) ) / (2.0*pi*sigma2^2)^1.5 -
                  exp( -r^2 / (2.0*sigma1^2) ) / (2.0*pi*sigma1^2)^1.5
    end
    deltaV = FD.hx * FD.hy * FD.hz
    Laplacian3d = build_nabla2_matrix( FD, func_1d=build_D2_matrix_9pt )
    prec = aspreconditioner(ruge_stuben(Laplacian3d))
    @printf("Test norm charge: %18.10f\n", sum(rho)*deltaV)
    print("Solving Poisson equation:\n")
    phi = Poisson_solve_PCG( Laplacian3d, prec, -4*pi*rho, 1000, verbose=true, TOL=1e-10 )
    # Calculation of Hartree energy
    Unum = 0.5*sum( rho .* phi ) * deltaV
    Uana = ((1.0/sigma1 + 1.0/sigma2 )/2.0 - sqrt(2.0)/sqrt(sigma1^2 + sigma2^2))/sqrt(pi)
    @printf("Numeric  = %18.10f\n", Unum)
    @printf("Uana     = %18.10f\n", Uana)
    @printf("abs diff = %18.10e\n", abs(Unum-Uana))
end
test_main([64,64,64])
\end{juliacode}