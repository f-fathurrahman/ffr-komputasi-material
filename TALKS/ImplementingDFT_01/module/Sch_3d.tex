\section{Schroedinger equation in 3d}

The 3d case of Schroedinger equation is a straightforward extension of the 2d case.
The Schroedinger equation thus reads:
\begin{equation}
\left[ -\frac{1}{2}\nabla^2 + V(x,y,z) \right] \psi(x,y,z) = E\,\psi(x,y,z)
\label{eq:sch_3d}
\end{equation}
%
where $\nabla^2$ is the Laplacian operator:
\begin{equation}
\nabla^2 = \frac{\partial^2}{\partial x^2} + \frac{\partial^2}{\partial y^2} + \frac{\partial^2}{\partial z^2}
\end{equation}

We begin by defining a struct called \txtinline{FD3dGrid} which is a
straightforward generalization of \txtinline{FD2dGrid}. The implementation of this
struct can be found in the file \txtinline{FD3d/FD3dGrid.jl}.

The Lagrangian operator in 3d also can be implemented by straightforward extension
of 2d case.
\begin{juliacode}
const ⊗ = kron

function build_nabla2_matrix( fdgrid::FD3dGrid; func_1d=build_D2_matrix_3pt )

    D2x = func_1d(fdgrid.Nx, fdgrid.hx)
    D2y = func_1d(fdgrid.Ny, fdgrid.hy)
    D2z = func_1d(fdgrid.Nz, fdgrid.hz)

    IIx = speye(fdgrid.Nx)
    IIy = speye(fdgrid.Ny)
    IIz = speye(fdgrid.Nz)

    ∇² = D2x⊗IIy⊗IIz + IIx⊗D2y⊗IIz + IIx⊗IIy⊗D2z 

    return ∇²
end
\end{juliacode}

The main difference is that we have used the symbol \txtinline{⊗} in place
of \txtinline{kron} function to make our code simpler.

We hope that at this point you will have no difficulties to create your own
3d Schroedinger equation solver.

Analytic solution for energy:
\begin{equation}
E_{n_{x} + n_{y} + n_{z}} = \hbar \omega \left( n_{x} + n_{y} + n_{z} + \frac{3}{2} \right)
\end{equation}

Degeneracies:
\begin{equation}
g_{n} = \frac{(n + 1)(n + 2)}{2}
\end{equation}

\begin{textcode}
n = n_x + n_y + n_z

n = 0: (1)(2)/2 = 1
n = 1: (2)(3)/2 = 3
n = 2: (3)(4)/2 = 6
\end{textcode}