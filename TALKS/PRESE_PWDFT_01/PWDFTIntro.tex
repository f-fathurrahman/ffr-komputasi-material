\documentclass[english,9pt]{beamer}

\usepackage{amsmath} % load this before unicode-math
\usepackage{amssymb}
%\usepackage{unicode-math}

\usepackage{fontspec}
\setmonofont{DejaVu Sans Mono}
%\setmathfont{STIXMath}
%\setmathfont{TeX Gyre Termes Math}

\usefonttheme[onlymath]{serif}

\setlength{\parskip}{\smallskipamount}
\setlength{\parindent}{0pt}

\setbeamersize{text margin left=5pt, text margin right=5pt}

\usepackage{amsmath}
\usepackage{amssymb}
\usepackage{braket}

\usepackage{minted}
\newminted{julia}{breaklines,fontsize=\scriptsize,texcomments=true}
\newminted{python}{breaklines,fontsize=\scriptsize,texcomments=true}
\newminted{bash}{breaklines,fontsize=\scriptsize,texcomments=true}
\newminted{text}{breaklines,fontsize=\scriptsize,texcomments=true}

\newcommand{\txtinline}[1]{\mintinline[fontsize=\scriptsize]{text}{#1}}
\newcommand{\jlinline}[1]{\mintinline[fontsize=\scriptsize]{julia}{#1}}

\definecolor{mintedbg}{rgb}{0.95,0.95,0.95}
\usepackage{mdframed}

%\BeforeBeginEnvironment{minted}{\begin{mdframed}[backgroundcolor=mintedbg]}
%\AfterEndEnvironment{minted}{\end{mdframed}}

\setcounter{secnumdepth}{3}
\setcounter{tocdepth}{3}

\makeatletter

 \newcommand\makebeamertitle{\frame{\maketitle}}%
 % (ERT) argument for the TOC
 \AtBeginDocument{%
   \let\origtableofcontents=\tableofcontents
   \def\tableofcontents{\@ifnextchar[{\origtableofcontents}{\gobbletableofcontents}}
   \def\gobbletableofcontents#1{\origtableofcontents}
 }

\makeatother

\usepackage{babel}

\begin{document}


\title{\texttt{PWDFT.jl}: Density Functional Theory Calculations with Julia}
\subtitle{Introduction and Current Status}
\author{Fadjar Fathurrahman}
\institute{
Engineering Physics Department \\
Research Center for Nanoscience and Nanotechnology \\
Institut Teknologi Bandung
}
\date{24 September 2019}


\frame{\titlepage}

\begin{frame}
\frametitle{Software packages for DFT calculations}

There are a lot of software packages for DFT calculations, for examples:
\begin{itemize}
\item Quantum ESPRESSO
\item VASP
\item ABINIT
\item Gaussian series: G03, G09, G16
\item NWchem
\end{itemize}

More extensive list:
{\scriptsize
\url{https://en.wikipedia.org/wiki/List_of_quantum_chemistry_and_solid-state_physics_software}
}

\end{frame}


\begin{frame}
\frametitle{Problems}

\begin{itemize}
\item These packages are very helpful for doing various calculations based on DFT.
%
These packages are suitable for black-box-type calculations where we are only concerned about the results.
%
\item However they are generally rather difficult to extend.
  \begin{itemize}
  \item New development on the DFT functionals:
  users generally need to wait for the next release of the package to use them
  (if these functionals are to be implemented at all).
  \item Custom calculations or post-processing steps:
  Users need to know in some detail
  about how the data they are interested in is represented or implemented in the package's source code.
  \end{itemize}
\end{itemize}

\end{frame}

\begin{frame}
\frametitle{Introducing PWDFT.jl}

\begin{itemize}
\item A package (more or less like a library, no executable) for electronic structure calculations
based on DFT.
\item Available at {\scriptsize\url{https://github.com/f-fathurrahman/PWDFT.jl}}
\item Using plane wave basis functions.
\item Implemented using Julia programming language.
\item My latest attempt to implement DFT softwares, previous attempts:
\begin{itemize}
  \item \txtinline{ffr-LFDFT}: {\scriptsize\url{https://github.com/f-fathurrahman/ffr-LFDFT}},
  using Lagrange basis functions, implemented in Fortran.
  \item \txtinline{ffr-PWDFT}: {\scriptsize\url{https://github.com/f-fathurrahman/ffr-PWDFT}}
  also using plane wave basis functions, implemented in Fortran.
\end{itemize}
\item Starting point is my Julia implementation of the code by Prof. Tomas Arias described in
his Practical DFT course. I extended the code to handle nonlocal pseudopotentials and
multiple k-points.
\item Taking many ideas from Quantum ESPRESSO, KSSOLV, ELK, etc.
\item No logo yet.
\end{itemize}

\end{frame}


\begin{frame}
\frametitle{Why I write another DFT package?}

Writing a DFT package from scratch is not the solution for all
problems.

[Rant mode ON]

Several questions appeared during my early days with DFT:
\begin{itemize}
\item Why my calculations are slow? What makes my calculations slow?
\item Is this slowness justified? How can I make it faster?
\item What are actually calculated by the DFT packages?
\end{itemize}

Learn by doing: how DFT or Kohn-Sham equations are solved (beyond
described in textbooks or research papers)

Frustation when trying to extend functionalities of available packages.

Educational purpose: the secret art of writing a DFT code is not yet documented
extensively in a book.

\end{frame}

\begin{frame}
\frametitle{Programming languages for DFT}
    
Programming languages used:
\begin{itemize}
\item Fortran and/or C/C++: ABINIT, VASP, Quantum Espresso, ...
\item Python: GPAW
\item MATLAB: KSSOLV, RESCU
\end{itemize}
    
Static languages: Fortran, C/C++

Dynamic languages: Python and MATLAB

\end{frame}



\begin{frame}
\frametitle{Julia programming language}

A rather new programming language (2012)

Syntax is familar to MATLAB or Python users

support for multidimensional array and linear algebra

Loop is fast!

\end{frame}


\begin{frame}
\frametitle{Aims of PWDFT.jl}

\begin{itemize}
\item Friendly-to-developers DFT package: enables quick implementation of various algorithms
\item educational purpose: simple yet powerful enough to carry out practical DFT calculations
for molecular and crystalline systems.
\end{itemize}

\end{frame}

\begin{frame}[fragile]
\frametitle{Julia installation}

I am assuming familiarity with command line.

Download Julia the current stable of Julia at
{\scriptsize\url{https://julialang.org/downloads}}
for your operating system.

Unpack the tarball. Usually after unpacking the tarball you should see a new directory
created: \txtinline{julia-1.x.x}, where \txtinline{1.x.x} is the version of Julia.

\begin{textcode}
julia-1.x.x/
├── bin
├── etc
├── include
├── lib
├── libexec
├── LICENSE.md
└── share
\end{textcode}

The Julia binary resides within the \txtinline{bin} directory. You can launch Julia interpreter
by executing the \txtinline{bin/julia} executable. After executing the binary you can see the
the following output:
\begin{textcode}
                _
    _       _ _(_)_     |  Documentation: https://docs.julialang.org
   (_)     | (_) (_)    |
    _ _   _| |_  __ _   |  Type "?" for help, "]?" for Pkg help.
   | | | | | | |/ _` |  |
   | | |_| | | | (_| |  |  Version 1.3.1 (2019-12-30)
  _/ |\__'_|_|_|\__'_|  |  Official https://julialang.org/ release
 |__/                   |
 
 julia>
\end{textcode}

\end{frame}



\begin{frame}[fragile]
\frametitle{Title}


\end{frame}



\input{Examples}

\input{SCF_related}

\begin{frame}
\frametitle{Direct energy minimization}

For systems with band gaps:

Conjugate gradient

Direct minimization

\end{frame}

\end{document}

