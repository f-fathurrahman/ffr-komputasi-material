\begin{frame}[fragile]
\frametitle{Julia installation}

I am assuming familiarity with command line.

Download Julia the current stable of Julia at
{\scriptsize\url{https://julialang.org/downloads}}
for your operating system.

Unpack the tarball. Usually after unpacking the tarball you should see a new directory
created: \txtinline{julia-1.x.x}, where \txtinline{1.x.x} is the version of Julia.

\begin{textcode}
julia-1.x.x/
├── bin
├── etc
├── include
├── lib
├── libexec
├── LICENSE.md
└── share
\end{textcode}

The Julia binary resides within the \txtinline{bin} directory. You can launch Julia interpreter
by executing the \txtinline{bin/julia} executable. After executing the binary you can see the
the following output:
\begin{textcode}
                _
    _       _ _(_)_     |  Documentation: https://docs.julialang.org
   (_)     | (_) (_)    |
    _ _   _| |_  __ _   |  Type "?" for help, "]?" for Pkg help.
   | | | | | | |/ _` |  |
   | | |_| | | | (_| |  |  Version 1.3.1 (2019-12-30)
  _/ |\__'_|_|_|\__'_|  |  Official https://julialang.org/ release
 |__/                   |
 
 julia>
\end{textcode}

\end{frame}



\begin{frame}[fragile]
\frametitle{Title}


\end{frame}

