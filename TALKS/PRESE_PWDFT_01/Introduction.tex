\begin{frame}
\frametitle{Software packages for DFT calculations}

There are a lot of software packages for DFT calculations, for examples:
\begin{itemize}
\item Quantum ESPRESSO
\item VASP
\item ABINIT
\item Gaussian series: G03, G09, G16
\item NWchem
\end{itemize}

More extensive list:
{\scriptsize
\url{https://en.wikipedia.org/wiki/List_of_quantum_chemistry_and_solid-state_physics_software}
}

\end{frame}


\begin{frame}
\frametitle{Problems}

\begin{itemize}
\item These packages are very helpful for doing various calculations based on DFT.
%
These packages are suitable for black-box-type calculations where we are only concerned about the results.
%
\item However they are generally rather difficult to extend.
  \begin{itemize}
  \item New development on the DFT functionals:
  users generally need to wait for the next release of the package to use them
  (if these functionals are to be implemented at all).
  \item Custom calculations or post-processing steps:
  Users need to know in some detail
  about how the data they are interested in is represented or implemented in the package's source code.
  \end{itemize}
\end{itemize}

\end{frame}

\begin{frame}
\frametitle{Introducing PWDFT.jl}

\begin{itemize}
\item A package (more or less like a library, no executable) for electronic structure calculations
based on DFT.
\item Available at {\scriptsize\url{https://github.com/f-fathurrahman/PWDFT.jl}}
\item Using plane wave basis functions.
\item Implemented using Julia programming language.
\item My latest attempt to implement DFT softwares, previous attempts:
\begin{itemize}
  \item \txtinline{ffr-LFDFT}: {\scriptsize\url{https://github.com/f-fathurrahman/ffr-LFDFT}},
  using Lagrange basis functions, implemented in Fortran.
  \item \txtinline{ffr-PWDFT}: {\scriptsize\url{https://github.com/f-fathurrahman/ffr-PWDFT}}
  also using plane wave basis functions, implemented in Fortran.
\end{itemize}
\item Starting point is my Julia implementation of the code by Prof. Tomas Arias described in
his Practical DFT course. I extended the code to handle nonlocal pseudopotentials and
multiple k-points.
\item Taking many ideas from Quantum ESPRESSO, KSSOLV, ELK, etc.
\item No logo yet.
\end{itemize}

\end{frame}


\begin{frame}
\frametitle{Why I write another DFT package?}

Writing a DFT package from scratch is not the solution for all
problems.

[Rant mode ON]

Several questions appeared during my early days with DFT:
\begin{itemize}
\item Why my calculations are slow? What makes my calculations slow?
\item Is this slowness justified? How can I make it faster?
\item What are actually calculated by the DFT packages?
\end{itemize}

Learn by doing: how DFT or Kohn-Sham equations are solved (beyond
described in textbooks or research papers)

Frustation when trying to extend functionalities of available packages.

Educational purpose: the secret art of writing a DFT code is not yet documented
extensively in a book.

\end{frame}