\documentclass[english,9pt]{beamer}

\usepackage{amsmath} % load this before unicode-math
\usepackage{amssymb}
%\usepackage{unicode-math}

\usepackage{fontspec}
\setmonofont{DejaVu Sans Mono}
%\setmathfont{STIXMath}
%\setmathfont{TeX Gyre Termes Math}

\usefonttheme[onlymath]{serif}

\setlength{\parskip}{\smallskipamount}
\setlength{\parindent}{0pt}

\setbeamersize{text margin left=5pt, text margin right=5pt}

\usepackage{amsmath}
\usepackage{amssymb}
\usepackage{braket}

\usepackage{minted}
\newminted{julia}{breaklines,fontsize=\scriptsize,texcomments=true}
\newminted{python}{breaklines,fontsize=\scriptsize,texcomments=true}
\newminted{bash}{breaklines,fontsize=\scriptsize,texcomments=true}
\newminted{text}{breaklines,fontsize=\scriptsize,texcomments=true}

\newcommand{\txtinline}[1]{\mintinline[fontsize=\scriptsize]{text}{#1}}
\newcommand{\jlinline}[1]{\mintinline[fontsize=\scriptsize]{julia}{#1}}

\definecolor{mintedbg}{rgb}{0.95,0.95,0.95}
\usepackage{mdframed}

%\BeforeBeginEnvironment{minted}{\begin{mdframed}[backgroundcolor=mintedbg]}
%\AfterEndEnvironment{minted}{\end{mdframed}}

\setcounter{secnumdepth}{3}
\setcounter{tocdepth}{3}

\makeatletter

 \newcommand\makebeamertitle{\frame{\maketitle}}%
 % (ERT) argument for the TOC
 \AtBeginDocument{%
   \let\origtableofcontents=\tableofcontents
   \def\tableofcontents{\@ifnextchar[{\origtableofcontents}{\gobbletableofcontents}}
   \def\gobbletableofcontents#1{\origtableofcontents}
 }

\makeatother

\usepackage{babel}

\begin{document}


\title{Introduction to Research Topics}
\subtitle{Ongoing and Planned}
\author{Fadjar Fathurrahman}
\institute{
Engineering Physics Department \\
Research Center for Nanoscience and Nanotechnology \\
Institut Teknologi Bandung
}
\date{}


\frame{\titlepage}


\begin{frame}
\frametitle{Overview}

The problem

My current research

General aspects

programming:

Topics for students:


\end{frame}


\begin{frame}
\frametitle{The problem: Solving Kohn-Sham problems}

What is it? 

Where 
\begin{itemize}
\item We are trying to find what are really computed behind the software packages such
 as Quantum ESPRESSO, VASP, ABINIT, Gaussian, etc.
\item We are trying to implement (write codes) that solve the Kohn-Sham problem.
\end{itemize}

\end{frame}


\begin{frame}
\frametitle{The Kohn-Sham energy functional}

Total energy:
\begin{eqnarray}
E\left[\{\psi_{i}(\mathbf{r})\}\right] =
-\frac{1}{2} \int \psi_{i}(\mathbf{r}) \nabla^{2} \psi_{i}(\mathbf{r})\,\mathrm{d}\mathbf{r} +
\int \rho(\mathbf{r}) V_{\mathrm{ext}}(\mathbf{r})\,\mathrm{d}\mathbf{r} + \\
\frac{1}{2}\int
\frac{\rho(\mathbf{r}) \rho(\mathbf{r}')}{\left|\mathbf{r}-\mathbf{r}'\right|}\,
\mathrm{d}\mathbf{r}\,\mathrm{d}\mathbf{r}' + E_{\mathrm{xc}}\left[\rho(\mathbf{r})\right]
\end{eqnarray}

Electron density
\begin{equation}
\rho(\mathbf{r}) = \sum_{i} f_{i}\, \psi_{i}^{*}(\mathbf{r}) \psi_{i}(\mathbf{r})
\end{equation}

External potential:
\begin{equation}
V_{\mathrm{ext}}(\mathbf{r}) = \sum_{I} \frac{Z_{I}}{\left|\mathbf{r} - \mathbf{R}_{I}\right|}
\end{equation}

\end{frame}



\begin{frame}
\frametitle{Kohn-Sham equation}

\begin{equation}
\hat{H}_{\mathrm{KS}}\,\psi_{i}(\mathbf{r}) = \epsilon_{i}\,\psi_{i}(\mathbf{r})
\end{equation}

\begin{equation}
\hat{H}_{\mathrm{KS}} = -\frac{1}{2}\nabla^{2} + V_{\mathrm{ext}}(\mathbf{r}) +
V_{\mathrm{Ha}}(\mathbf{r}) + V_{\mathrm{xc}}(\mathbf{r})
\end{equation}

\begin{equation}
V_{\mathrm{Ha}}(\mathbf{r}) = \int
\frac{\rho(\mathbf{r})}{\mathbf{r} - \mathbf{r}'}\,\mathrm{d}\mathbf{r}'
\end{equation}

\begin{equation}
\nabla^2 V_{\mathrm{Ha}}(\mathbf{r}) = -4\pi\rho(\mathbf{r})
\end{equation}

\begin{equation}
V_{\mathrm{xc}}(\mathbf{r}) = \frac{\delta E[\rho(\mathbf{r})]}{\delta \rho(\mathbf{r})}
\end{equation}

\end{frame}


\begin{frame}
\frametitle{Various ways to "discretize" Kohn-Sham problem}

\begin{itemize}
\item Local basis set: GTO (Gaussian program, NWCHEM, ORCA, CP2K, etc),
  ETO and numerical basis (fhi-aims, SIESTA, OPENMX, etc),
  muffin-tin orbitals (LMTO, RPSt, ..)
\item Plane wave basis (QE, VASP, ABINIT, JDFTx, etc)
\item Real-space based: finite-difference (Octopus, GPAW, PARSEC), finite-element (HelFEM, )
\item Mixed and augmented basis set: FLAPW (Wien2k, ELK, exciting, fleur),
  PW + AO (Tombo)
\end{itemize}

\end{frame}


\begin{frame}[plain]

\includegraphics[scale=1.0]{images/page4.pdf}

\end{frame}


\begin{frame}
\frametitle{PWDFT.jl}

\begin{itemize}
\item One of my many attempts to implement a Kohn-Sham DFT solver
(other attempts can be found in 
{\footnotesize\url{https://github.com/f-fathurrahman}}).
\item Using plane wave basis set and pseudopotentials (currently only GTH pseudopotentials
are supported)
\item LDA VWN + GGA PBE
\item SCF (with density or potential mixing) and direct-minimization (for semiconductor)
\item Force calculation is implemented but not tested yet
\end{itemize}

\end{frame}


\begin{frame}
\frametitle{Topics related to improving PWDFT.jl}

There are many topics:

\begin{itemize}
\item Documentation, testing, code clean up, visualization
\item Adding and testing new Kohn-Sham solvers: Chebyshev-filtered
subspace iteration, direct-minimization for metals, new mixing schemes
\item Parallelization: multithreading, MPI and GPU (using CUDA)
\item Implementing geometry optimization and molecular dynamics
\item Implement calculation of stress tensor
\item New physics: implement exact-exchange, meta GGA functionals, and vdW-DF
\item Implement USPP and PAW
\end{itemize}


\end{frame}



\begin{frame}
\frametitle{Machine learning related}

Goals: replacing time-consuming quantum mechanical calculation with faster calculation
via learning from training data.

\begin{itemize}
\item force-field related
\item lsl
\end{itemize}

Neural-networkw

Kernel regression

Most of the tools are in rapid development.

Strongly recommended to write or implement your own tools.

\end{frame}


\begin{frame}
\frametitle{General tips + workflows}

read softwares (need Fortran (both modern and F77), C/C++, Python, MATLAB, ...)

learn how they work

think + read books + how to improve

rewrite them using your own language, add features ...

\end{frame}




\end{document}

