\documentclass[b5paper,11pt,english]{article} % screen setting
\usepackage[b5paper]{geometry}

%\documentclass[b5paper,11pt,bahasa]{article} % screen setting
%\usepackage[b5paper]{geometry}

\geometry{verbose,tmargin=1.5cm,bmargin=1.5cm,lmargin=1.5cm,rmargin=1.5cm}

\setlength{\parskip}{\smallskipamount}
\setlength{\parindent}{0pt}

%\usepackage{cmbright}
%\renewcommand{\familydefault}{\sfdefault}

\usepackage[libertine]{newtxmath}
\usepackage[no-math]{fontspec}

\setmainfont{Linux Libertine O}

\setmonofont{JuliaMono-Regular}


\usepackage{hyperref}
\usepackage{url}
\usepackage{xcolor}

\usepackage{amsmath}
\usepackage{amssymb}

\usepackage{graphicx}
\usepackage{float}

\usepackage{minted}

\newminted{julia}{breaklines,fontsize=\scriptsize}
\newminted{python}{breaklines,fontsize=\scriptsize}

\newminted{bash}{breaklines,fontsize=\scriptsize}
\newminted{text}{breaklines,fontsize=\scriptsize}

\newcommand{\txtinline}[1]{\mintinline[breaklines,fontsize=\scriptsize]{text}{#1}}
\newcommand{\jlinline}[1]{\mintinline[breaklines,fontsize=\scriptsize]{julia}{#1}}
\newcommand{\pyinline}[1]{\mintinline[breaklines,fontsize=\scriptsize]{python}{#1}}

\newmintedfile[juliafile]{julia}{breaklines,fontsize=\scriptsize}
\newmintedfile[pythonfile]{python}{breaklines,fontsize=\scriptsize}
% f-o-otnotesize

\usepackage{mdframed}
\usepackage{setspace}
\onehalfspacing

\usepackage{appendix}

\newcommand{\highlighteq}[1]{\colorbox{blue!25}{$\displaystyle#1$}}
\newcommand{\highlight}[1]{\colorbox{red!25}{#1}}

\definecolor{mintedbg}{rgb}{0.95,0.95,0.95}
\BeforeBeginEnvironment{minted}{
    \begin{mdframed}[backgroundcolor=mintedbg,%
        topline=false,bottomline=false,%
        leftline=false,rightline=false]
}
\AfterEndEnvironment{minted}{\end{mdframed}}

% -------------------------
\begin{document}

\title{%
LAPW Notes
}
\author{}
\date{}
\maketitle

Radial Schroedinger equation:
\begin{equation}
\left[ -\frac{1}{2}
\frac{\mathrm{d}^2}{\mathrm{d}r^2} +
\frac{l(l+1)}{2r^2} + V_{\mathrm{KS}}(r) - \epsilon_{nl}
\right] u_{nl}(r) = 0
\end{equation}

Numerov integration
\begin{equation}
x''(t) = f(t) x(t) + u(t)
\end{equation}


\begin{equation}
x(h) + x(-h) = 2x(0) + h^2\left[ f(0)x(0) + u(0) \right] +
\frac{2}{4!}h^4 x^{(4)}(0) + \mathcal{O}(h^6)
\end{equation}
For $\mathcal{O}(h^4)$ algorithm, we can neglect $x^{(4)}$ term and get
the following:
\begin{equation}
x_{i+1} - 2x_{i} + x_{i+1} = h^2 ( f_{i} x_{i} + u_{i} )
\end{equation}

From the differential equation we can get:
\begin{equation}
x^{(4)} = \frac{\mathrm{d}^2 x''(t)}{\mathrm{d}t^2} =
\frac{\mathrm{d}^2}{\mathrm{d}t^2} ( f(t) x(t) + u(t) ) 
\end{equation}
which can be approximated by:
\begin{equation}
x^{(4)} \approx \frac{f_{i+1} x_{i+1} + u_{i+1} - 2f_{i} x_{i} - 2u_{i} + f_{i-1}x_{i-1} + u_{i-1}}{h^2}
\end{equation}
Using this we can write:
\begin{align}
x_{i+1} - 2x_{i} + x_{i-1} = & h^2 \left( f_{i}x_{i} + u_{i} \right) + \\
& \frac{h^2}{12} \left( f_{i+1}x_{i+1} + u_{i+1} - 2f_{i}x_{i} - 2u_{i} +
f_{i-1}x_{i-1} + u_{i-1}
\right)
\end{align}
Using $w_{i} = x_{i}\left( 1 - \frac{h^2}{12}f_{i} \right) - \frac{h^2}{12}u_{i}$
we have the following scheme:
\begin{equation}
w_{i+1} - 2w_{i} + w_{i-1} = h^2 \left( f_{i} x_{i} + u_{i} \right) + 
\mathcal{O}(h^6)
\end{equation}

One needs to solve Schroedinger equation for the bound states $\epsilon_{nl}$.
We start solving Schroedinger equation far from the nucleus to avoid errors due to
possible oscillations of $u_{nl}(v)$ close to nucleus.
A good starting guess comes from the solution of hydrogen atom:
\begin{align}
u(R_{\mathrm{max}}) = R_{\mathrm{max}}e^{-ZR_{\mathrm{max}}}
\end{align}


\end{document}
