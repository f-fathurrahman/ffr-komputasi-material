\documentclass[a4paper,11pt]{article} % screen setting

\usepackage[a4paper]{geometry}
%\geometry{verbose,tmargin=1.5cm,bmargin=1.5cm,lmargin=1.5cm,rmargin=7.5cm}

\setlength{\parskip}{\smallskipamount}
\setlength{\parindent}{0pt}

\usepackage{minted}

\newminted{julia}{breaklines}
\newminted{python}{breaklines}

\newminted{bash}{breaklines}
\newminted{text}{breaklines}

\newcommand{\txtinline}[1]{\mintinline[breaklines]{text}{#1}}
\newcommand{\jlinline}[1]{\mintinline[breaklines]{julia}{#1}}
\newcommand{\pyinline}[1]{\mintinline[breaklines]{python}{#1}}


\usepackage{setspace}
\onehalfspacing

\begin{document}


\title{Ujian Tengah Semester \\
TF4063 Sains Data dan Rekayasa}
\author{}
\date{13 Oktober 2023}
\maketitle


Kerjakan dengan menggunakan Jupyter Notebook. Jika Anda lebih memilih untuk menggunakan
script Python, sertakan juga penjelasan dalam bentuk file pdf.

Gabungkan semua berkas yang digunakan dalam satu folder dengan format
\texttt{NIM\_Nama} dan dikompresi dalam satu file zip dengan format
\texttt{NIM\_Nama.zip}.

Jadikan NIM Anda sebagai \textit{seed} untuk pembangkit bilangan acak
(\textit{random number generator}) yang digunakan.

Waktu pengerjaan adalah \emph{satu minggu}: dikumpulkan paling lambat
tanggal 20 Oktober 2023.

\begin{enumerate}
%
%
\item Lakukan regresi linear pada data sintetik yang Anda buat dengan
menggunakan polinomial orde-10 (Anda dapat menentukan sendiri koefisien
polinomial dan amplitudo \textit{noise} yang Anda gunakan).
Gunakan kode yang Anda buat sendiri (hanya menggunakan Numpy)
untuk mengimplementasikan
solusi analitik dari regresi linear dan dengan menggunakan sklearn.
Apakah implementasi yang Anda buat memberikan
koefisien yang sama dengan hasil dari sklearn? Bandingkan hasil yang diperoleh
jika Anda menggunakan standardisasi fitur dan tanpa standardisasi fitur. Jelaskan
hasil-hasil yang Anda peroleh.
%
%
\item Lakukan regresi linear dengan menggunakan sklearn (kelas
\pyinline{LinearRegression}) pada dataset
\texttt{aqsoldb} menggunakan seluruh data. Buat plot paritas dan tampilkan
koefisien determinasi $R^{2}$ dan hitung RMSE (\textit{root mean square error})
yang diperoleh.
%
%
\item Baca dokumentasi mengenai kelas \texttt{MLPRegressor} pada sklearn.
Gunakan kelas ini untuk melakukan regresi pada dataset \texttt{aqsoldb}.
Buat plot paritas dan tampilkan
koefisien determinasi yang diperoleh. Bandingkan hasil yang Anda peroleh pada soal
sebelumnya (menggunakan model linear).
Catatan:
  \begin{itemize}
  \item Coba dapatkan hasil yang lebih baik daripada model linear dengan
  cara, misalnya, mengubah-ubah parameter \pyinline{hidden_layer_sizes}.
  \item Gunakan fitur dan \textit{preprocessing} yang sama seperti pada
  soal sebelumnya.
  \end{itemize}
%
%
\item Pada dua soal sebelumnya, Anda menggunakan seluruh dataset yang ada pada
proses pelatihan atau \textit{fitting}.
Pada soal ini, Anda akan melakukan perbandingan antara model linear
(\pyinline{LinearRegression}) dan nonlinear (\pyinline{MLPRegressor}) pada
dataset yang sudah dibagi menjadi data latih dan data uji. Gunakan
fungsi \pyinline{train_test_split} pada sklearn untuk membagi dataset menjadi
data latih dan data uji. Anda dapat menentukan sendiri ukuran dan data uji dan
data latih yang digunakan. Lakukan proses yang sama seperti pada soal sebelumnya,
namun sekarang plot paritas, $R^{2}$ dan RMSE ditampilkan untuk
masing-masing untuk data latih dan data uji.
Catatan: Pastikan data uji dan data latih yang digunakan
untuk \pyinline{LinearRegression} dan \pyinline{MLPRegressor} adalah sama.
%
%
\item Berikan penjelasan mengenai mengenai validasi silang $k$-lipatan
($k$\textit{-fold cross validation})
dan buat implementasinya pada Python (implementasi yang Anda buat sendiri,
tanpa menggunakan pustaka khusus seperti pada sklearn).
Aplikasikan pada kasus pemilihan orde polinomial yang tepat pada
data sintetik.
%
%
\item Salah satu metode yang sering digunakan untuk mengatasi
\textit{overfitting} adalah regularisasi. Pada model linear, regularisasi
dapat dilakukan dengan cara memodifikasi fungsi rugi menjadi
fungsi rugi
$$
\mathcal{L}' = \mathcal{L} + \lambda \mathbf{w}^{\mathsf{T}}\mathbf{w}
$$
dengan $\lambda$ adalah suatu parameter dan $\mathcal{L}$
adalah fungsi rugi biasa (tanpa regularisasi).
Turunkan ekpresi dari $\mathbf{w}$ yang dapat meminimumkan $\mathcal{L}'$.
Implementasikan fungsi untuk menghitung $\mathbf{w}$ pada Python
dan berikan contoh penggunaannya pada data sintetik polinomial.
Referensi: Rogers dan Girolami.
\textit{A First Course in Machine Learning (2nd Ed)}.
CRC Press. 2017.
%
%
\item Cari \textit{dataset} di internet (baik dari Kaggle, artikel ilmiah,
ataupun sumber lain) untuk masalah regresi dan aplikasikan model linear
dan non-linear (boleh menggunakan
model nonlinear selain \pyinline{MLPRegressor}). Berikan penjelasan singkat
mengenai \textit{dataset} yang dipilih, misalnya sumber data (jika ada),
penjelasan mengenai kolom data, dan sebagainya.
Pisahkan data menjadi data latih dan data uji, kemudian tampilkan
performa dari model yang dilatih (dapat menggunakan plot paritas, koefisien
determinasi, dan/atau metrik lainnya). 
\end{enumerate}

\end{document}
