\documentclass[a4paper,11pt]{article} % screen setting

\usepackage[a4paper]{geometry}
%\geometry{verbose,tmargin=1.5cm,bmargin=1.5cm,lmargin=1.5cm,rmargin=7.5cm}

\setlength{\parskip}{\smallskipamount}
\setlength{\parindent}{0pt}

\usepackage{minted}

\newminted{julia}{breaklines}
\newminted{python}{breaklines}

\newminted{bash}{breaklines}
\newminted{text}{breaklines}

\newcommand{\txtinline}[1]{\mintinline[breaklines]{text}{#1}}
\newcommand{\jlinline}[1]{\mintinline[breaklines]{julia}{#1}}
\newcommand{\pyinline}[1]{\mintinline[breaklines]{python}{#1}}

\usepackage{hyperref}

\usepackage{setspace}
\onehalfspacing

\hyphenation{li-ne-ar}
\hyphenation{mi-sal-nya}
\hyphenation{va-li-da-ti-on}
\hyphenation{re-gu-la-ri-sa-si}
\hyphenation{ge-ne-ra-tor}

\begin{document}


\title{Ujian Akhir Semester \\
TF4063 Sains Data dan Rekayasa}
\author{}
\date{15 Desember 2023}
\maketitle


Kerjakan dengan menggunakan Jupyter Notebook.
Jika Anda lebih memilih untuk menggunakan
script Python, sertakan juga penjelasan dalam bentuk file pdf.

Anda dapat mengerjakan tugas ini secara individu atau berkelompok (dua atau tiga orang).
Jika pengerjaan dilakukan secara berkelompok, berikan informasi mengenai kontribusi dari
masing-masing anggota.

Gabungkan semua berkas yang digunakan dalam satu folder dengan format\\
\texttt{NIM1\_NIM2\_NIM3\_NamaNamaAnggota}\\
dan dikompresi dalam satu file zip dengan format\\
\texttt{NIM1\_NIM2\_NIM3\_NamaNamaAnggota.zip}.

Jadikan salah satu NIM sebagai \textit{seed} untuk pembangkit bilangan acak
(\textit{random number generator}) jika diperlukan.

Jawaban dikumpulkan pada MS Teams paling lambat tanggal 23 Desember 2023.

\begin{enumerate}
%
%
\item Jelaskan mengenai beberapa metrik-metrik yang sering digunakan dalam
klasifikasi dan berikan implementasi sederhana dari perhitungan
metrik tersebut dalam bentuk fungsi Python.
%
%
\item Jelaskan mengenai \textit{confusion matrix} dan
kurva \textit{receiver-operating characteristic} dalam konteks klasifikasi.
%
%
\item Salah satu dataset yang sering digunakan untuk menguji model klasifikasi
adalah dataset \textit{iris}.
Pada
scikit-learn Anda dapat menggunakan fungsi \pyinline{datasets.load_iris()}
atau Anda dapat menggunakan sumber lain untuk mendapatkan dataset ini.
Pilih hanya dua
kelas saja dari dataset tersebut dan gunakan model regresi logistik
pada Bab 4 dari \url{dmol.pub} untuk membangun model klasifikasi dari dua kelas data yang
Anda pilih. Lengkapi jawaban Anda dengan visualisasi dan penjelasan singkat mengenai
hasil yang Anda peroleh.
%
%
\item Salah satu dataset lain yang juga sering digunakan untuk menguji
model klasifikasi adalah \textit{mnist handwritten digits}. Pada
scikit-learn Anda dapat menggunakan fungsi \pyinline{datasets.load_digits()} atau
Anda juga dapat menggunakan sumber lain untuk mendapatkan dataset ini.
Ulangi soal sebelumnya
namun menggunakan dataset ini.
%
%
\item Buat data sintetik yang memiliki struktur klaster tertentu
jika divisualisasikan pada dua dimensi (ada dua dimensi atau fiture "penting").
Anda dapat memilih sendiri jumlah data yang Anda akan buat.
Kemudian tambahkan beberapa fitur acak lain, misalnya lima fitur sehingga
data yang Anda miliki sekarang memiliki, misalnya 7 fitur. Anda dapat menukar
kolom atau fitur secara acak sehingga fitur yang penting tidak lagi berada
pada dua fitur pertama.
  \begin{enumerate}
    \item Buat \textit{pairplot} dari data yang Anda buat. Berikan penjelasan
    mengenai \textit{pairplot} tersebut: apakah Anda dapat mengidentifikasi
    dimensi penting dari \textit{pairplot} tersebut?
    \item Lakukan prosedur \textit{principal component analysis} (PCA) pada
    data yang Anda buat dan visualisasikan nilai eigen yang Anda dapatkan
    \item Lakukan reduksi dimensi (atau proyeksi) pada data yang Anda peroleh
    dengan menggunakan beberapa eigenvektor dominan (yang memiliki
    nilai eigen terbesar) yang Anda peroleh dari PCA. Lakukan juga visualisasi
    pada data yang sudah tereduksi ini.
  \end{enumerate}
\end{enumerate}

\end{document}



