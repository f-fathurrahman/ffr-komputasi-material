\documentclass[a4paper,11pt]{article} % screen setting

\usepackage[a4paper]{geometry}
\geometry{verbose,tmargin=1.5cm,bmargin=1.5cm,lmargin=1.5cm,rmargin=1.5cm}

\setlength{\parskip}{\smallskipamount}
\setlength{\parindent}{0pt}

\usepackage{amsmath}
\usepackage{graphicx}

\usepackage{minted}

\newminted{julia}{breaklines}
\newminted{python}{breaklines}

\newminted{bash}{breaklines}
\newminted{text}{breaklines}

\newcommand{\txtinline}[1]{\mintinline[breaklines]{text}{#1}}
\newcommand{\jlinline}[1]{\mintinline[breaklines]{julia}{#1}}
\newcommand{\pyinline}[1]{\mintinline[breaklines]{python}{#1}}

\usepackage{hyperref}

\usepackage{setspace}
\onehalfspacing

\hyphenation{li-ne-ar}
\hyphenation{mi-sal-nya}
\hyphenation{va-li-da-ti-on}
\hyphenation{re-gu-la-ri-sa-si}
\hyphenation{ge-ne-ra-tor}

\begin{document}


\title{Ujian Akhir Semester \\
TF4063 Sains Data dan Rekayasa}
\author{}
\date{}
\maketitle


Kerjakan dengan menggunakan Jupyter Notebook.
Jika Anda lebih memilih untuk menggunakan
script Python, sertakan juga penjelasan dalam bentuk file pdf.

Anda dapat mengerjakan tugas ini secara individu.

Gabungkan semua berkas yang digunakan dalam satu folder dengan format\\
\texttt{NIM\_Nama}\\
dan dikompresi dalam satu file zip dengan format\\
\texttt{NIM\_Nama.zip}.

Jadikan NIM Anda sebagai \textit{seed} untuk pembangkit bilangan acak
(\textit{random number generator}) jika diperlukan.

Jawaban dikumpulkan pada MS Teams paling lambat tanggal 16 Juni 2024.

\begin{enumerate}
%
%
\item Jelaskan mengenai beberapa metrik-metrik yang sering digunakan dalam
klasifikasi dan berikan implementasi sederhana dari perhitungan
metrik tersebut dalam bentuk fungsi Python.
%
%
\item Jelaskan mengenai \textit{confusion matrix} dan
kurva \textit{receiver-operating characteristic} dalam konteks klasifikasi.
%
%
\item Salah satu dataset yang sering digunakan untuk menguji model klasifikasi
adalah dataset \textit{iris}.
Pada
\pyinline{scikit-learn} Anda dapat menggunakan fungsi \pyinline{datasets.load_iris()}
atau Anda dapat menggunakan sumber lain untuk mendapatkan dataset ini.
Pilih hanya dua
kelas saja dari dataset tersebut dan gunakan model regresi logistik
pada Bab 4 dari \url{dmol.pub} untuk membangun model klasifikasi dari dua kelas data yang
Anda pilih. Lengkapi jawaban Anda dengan visualisasi dan penjelasan singkat mengenai
hasil yang Anda peroleh.
%
%
\item Salah satu dataset lain yang juga sering digunakan untuk menguji
model klasifikasi adalah \textit{mnist handwritten digits}. Pada
scikit-learn Anda dapat menggunakan fungsi \pyinline{datasets.load_digits()} atau
Anda juga dapat menggunakan sumber lain untuk mendapatkan dataset ini.
Ulangi soal sebelumnya namun menggunakan dataset ini.
%
%
\item Dataset untuk klasifikasi citra yang juga sering digunakan adalah
\pyinline{Fashion MNIST}. Dataset ini juga dapat diperoleh langsung
melalui TensorFlow dan Keras.
  \begin{enumerate}
    \item Jelaskan secara singkat mengenai dataset ini: ada berapa kelas
    yang tersedia, format atau ukuran data, dan sebagainya
    \item Buat model klasifikasi \textit{multiclass} dengan menggunakan
    jaringan saraf tiruan \textit{multilayer} pada data \pyinline{Fashion MNIST}.
    Jelaskan secara detil mengenai protokol atau langkah-langkah yang Anda gunakan untuk
    membuat model dan mengevaluasi performa dari model. Buat beberapa model (minimal tiga)
    dengan \textit{hyperparameter} yang berbeda dan bandingkan hasilnya.
  \end{enumerate}
%
%
\item Sama dengan soal sebelumnya, namun dengan menambahkan \textit{convolution layer}
dan jenis \textit{layer} lain (\pyinline{MaxPooling} atau yang lain) pada model Anda.
%
%
\item Buat data sintetik yang memiliki struktur klaster tertentu
jika divisualisasikan pada dua dimensi (ada dua dimensi atau fiture "penting").
Anda dapat memilih sendiri jumlah data yang Anda akan buat.
Kemudian tambahkan beberapa fitur acak lain, misalnya 5 fitur tambahan sehingga
data yang Anda miliki sekarang memiliki, misalnya 7 fitur. Anda dapat menukar
kolom atau fitur secara acak sehingga fitur yang penting tidak lagi berada
pada dua fitur pertama.
  \begin{enumerate}
    \item Buat \textit{pairplot} dari data yang Anda buat. Berikan penjelasan
    mengenai \textit{pairplot} tersebut: apakah Anda dapat mengidentifikasi
    dimensi penting dari \textit{pairplot} tersebut?
    \item Lakukan prosedur \textit{principal component analysis} (PCA) pada
    data yang Anda buat dan visualisasikan nilai eigen yang Anda dapatkan
    \item Lakukan reduksi dimensi (atau proyeksi) pada data yang Anda peroleh
    dengan menggunakan beberapa eigenvektor dominan (yang memiliki
    nilai eigen terbesar) yang Anda peroleh dari PCA. Lakukan juga visualisasi
    pada data yang sudah tereduksi ini.
  \end{enumerate}
%
%
\item Jelaskan mengenai \textit{singular value decomposition} (SVD):
\begin{equation*}
\mathbf{X} = \mathbf{U}\mathbf{\Sigma}\mathbf{V}^{*}
\end{equation*}
dan hubungannya dengan PCA.
%
%
\item Salah satu kegunaan dari SVD adalah untuk aproksimasi
\textit{rank}-rendah (\textit{low rank approximation}) yang
optimal terhadap suatu matriks $\mathbf{X}$. Aproksimasi ini
dapat ditulis sebagai:
\begin{equation}
\tilde{\mathbf{X}} = \sum_{k=1}^{r} \sigma_{k} \mathbf{u}_{k} \mathbf{v}^{*}_{k}
= \sigma_{1} \mathbf{u}_{1} \mathbf{v}^{*}_{1} +
\sigma_{2} \mathbf{u}_{2} \mathbf{v}^{*}_{2} + \cdots
\sigma_{r} \mathbf{u}_{r} \mathbf{v}_{r}
\label{eq:svd02}
\end{equation}
di mana $\tilde{\mathbf{X}}$ adalah aproksimasi rank-$r$ dari matriks
$\mathbf{X}$, $\sigma_{k}$ adalah nilai singular ke-$k$ dari $\mathbf{X}$,
$\mathbf{u}_{k}$ dan $\mathbf{v}_{k}$ adalah kolom
ke-$k$ dari matriks $\mathbf{U}$ dan $\mathbf{V}$.
%
Sebagai data untuk matriks $\mathbf{X}$ Anda dapat menggunakan (misalnya, Anda juga dapat
menggunakan gambar lain)
citra berikut \footnote{Sumber:
\url{https://cdn.idntimes.com/content-images/community/2022/08/esradvpwoaqeech-16ed39a1a24ecefe857d22d6e29c1e14-cd868cde743721fb2ddda744c2ef3684.jpg}}:

{\centering
\includegraphics[scale=0.5]{images/kucing01.png}
\par}

Untuk mengaplikasikan Persamaan \eqref{eq:svd02}, Anda perlu terlebih dahulu
mengubah citra awal menjadi \textit{grayscale}.
Variasikan nilai $r$ pada Persamaan \eqref{eq:svd02} dan buat plot citra aproksimasi
yang diperoleh
\end{enumerate}

\end{document}



