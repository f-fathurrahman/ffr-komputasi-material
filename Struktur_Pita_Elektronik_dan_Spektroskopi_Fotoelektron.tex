\documentclass[a4paper,bahasa]{paper}

\usepackage{amssymb}
\usepackage{amsmath}

\usepackage{setspace}
\onehalfspacing

\usepackage{babel}

\renewcommand{\imath}{\mathfrak{i}}
\newcommand{\emath}{\mathrm{e}}

\begin{document}

\title{Struktur Pita Elektronik dan Spektroskopi Fotoelektron}
\author{Fadjar Fathurrahman}
\date{}
\maketitle

\section{Pendahuluan}

Signifikansi struktur pita elektronik

\section{Energi dan fungsi gelombang Kohn-Sham}
%
Untuk suatu konfigurasi inti atom yang diberikan, DFT dapat menghitung
total energi dari sistem elektron, $E$, serta kerapatan elektron, $n(\mathbf{r})$,
pada keadaan dasar. Persamaan Kohn-Sham dapat dituliskan sebagai:
\begin{equation}
-\frac{1}{2}\nabla^2\phi_{i}(\mathbf{r}) + V_{\mathrm{tot}}(\mathbf{r})\phi(\mathbf{r})
= \varepsilon_{i} \phi_{i}(\mathbf{r})
\end{equation}



\end{document}
